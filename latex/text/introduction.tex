% !TeX spellcheck = en_GB
\chapter{Introduction}

\begin{itemize}
	\item Molecular biology has had a great impact on our standards of living. There are a ton of diseases out there, and by studying how they work, we can (a) learn how cells and organisms function, and (b) propose certain treatments. Example: use amyloid aggregation to fight cancer.
	
	\item One interesting disease is caused by the Yersinia genus. Yersinia pestis is the species that caused the plague and other Yersinia bacteria. Vaccinations are available and incidence is low (600 cases per year worldwide), so low clinical importance.
	
	\item However, interesting mode of action. Actin filaments are broken down, revealing small self-organising micropatterns of actin left. Studying this might lead to new fundamental knowledge of cell function. (Ultimate goal)
	
	\item Current goal: set up a microscope system to study this disease, marrying super resolution microscopy and polarisation microscopy.
	
	\item Super resolution (STED) was already set up and had polarisation capabilities, but were never used. My main contributions lie in figuring out what all components do, how to characterise them etc. Most of that will be found in the Background and Methods sections.
	
	\item Another big part of work was developing a new polarisation microscopy method (pSTED) to improve the angular resolution of a polarisation microscope. Relate pSTED to a figure in the Yersinia pattern paper.
	
\end{itemize}