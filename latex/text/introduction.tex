% !TeX spellcheck = en_GB
\chapter{Introduction}

\begin{itemize}
	\item Molecular biology has had a great impact on our standards of living. There are a ton of diseases out there, and by studying how they work, we can (a) learn how cells and organisms function, and (b) propose certain treatments. Example: use amyloid aggregation to fight cancer.
	
	\item One interesting disease is caused by the Yersinia genus. Yersinia pestis is a species that caused the plague, which killed a third of the people living in europe and the plague of justinian. Vaccinations are available and incidence is low (600 cases per year worldwide), so low clinical importance currently.
	
	\item However, interesting mode of action. Actin filaments are broken down, revealing small self-organising micropatterns of actin left. Studying this might lead to new fundamental knowledge of cell function. (Ultimate goal of this research programme.)
	
	\item Current goal: set up a microscope system to study this disease, marrying super resolution microscopy and polarisation microscopy.
	
	Microscopy - imaging structures at microscopic scales - is a wide and varied field. It is widely accepted to have began in the sixteen-hundreds, with Antoni van Leeuwenhoek's discovery of bacteria and other single-celled organisms \cite{VanZuylen1981}. After that, microscopes have been getting higher resolution, but as lenses got better and better, they were not the resolution bottleneck any more. Specifically, in 1873, Ernst Abbe determined that the best possible focus that a microscope can reach is limited by the wavelength of the light used \cite{Abbe1873}. This meant that the microscopes of the time were limited to a resolution of roughly 400 nm (assuming focused visible white light). It was long believed that the Abbe limit was a fundamental limit of nature, and that the only way around it was by using light of a different wavelength. This is one of the reasons for the development of electron microscopes \cite{Smith2008}, as quantum mechanics explains that accelerated electrons have much shorter wavelength than visible light. 
	
	Fluorescence microscopy is an invaluable tool in modern biology \cite{Danial2016}. Unlike other methods, it is able to tag specific protein species and other relevant molecules in the cell with a fluorescent label. There are thousands of small organic fluorescent labels, and more are being developed \cite{Zhang2002, Resch-Genger2008}. The introduction of GFP and other fluorescent proteins was a remarkable development in the field, as labels can now be genetically fused to proteins of interest \cite{Shaner2005, Matlashov2020}. There were over thirty thousand papers collected in the PubMed database that mentioned fluorescence in the year 2020 alone.
	
	\item Super resolution (STED) was already set up and had polarisation capabilities, but were never used. My main contributions lie in figuring out what all components do, how to characterise them etc. Most of that will be found in the Background and Methods sections.
	
	\item Another big part of work was developing a new polarisation microscopy method (pSTED) to improve the angular resolution of a polarisation microscope. Relate pSTED to a figure in the Yersinia pattern paper.
	
\end{itemize}