% !TeX spellcheck = en_GB
\chapter{Introduction}
Between 1900 and 2000, advances in public health have increased life expectancy by 25 years \cite{Bunker1994}. The Centers for Disease Control and Prevention (CDC) credits advances in molecular biology such as vaccinations, better prevention of infectious diseases, better treatment of coronary heart disease and stroke, and more \cite{CDC1999}. Another pillar of molecular biology is the rapid development of SARS-CoV2 vaccines in the current pandemic \cite{Sadoff2021, Polack2020}. There are still many incurable diseases in the world, but by studying how they work, we can learn how cells and organisms function, and propose new treatments. Studying one disease can also guide treatment of another, as was the case when research into amyloidoses (a group of diseases including Alzheimer's disease, Parkinson's disease and diabetes type II, characterised by toxic amyloid protein aggregation), spun off a new direction of research into cancer treatment \cite{Gallardo2016}.

This thesis is part of a research programme that studies Yersinia. Yersinia is a genus of bacteria that includes Y. pestis, a bacterial species responsible for the plague, estimated to have killed millions of people in the Byzantine empire in the 6th century CE, and again 50 million in Europe in the 14th century, which was at that point about one third of the entire population \cite{Zietz2004}. Vaccinations are now available and the incidence is low (less than 1000 cases per year worldwide) \cite{WHO2014}. 
While there is no direct clinical need for more research into Yersinia, it is still a fascinating subject of fundamental importance, since it has an interesting mode of action. Yersinia bacteria secrete toxins into the cell that break down the actin network \cite{Ono2017}. The actin cytoskeleton works like the bones and muscles of a cell and is therefore vital to all kinds of processes, including embryonic development, cancer metastasis, and the immune system \cite{molbio, Horwitz2003, Umeda2016, Barnat2017, Lin2017}. Furthermore, it has been discovered that the absence of large actin fibres reveals self-organising actin microstructures \cite{Fritzsche2017a}. Their presence was quite a surprise and studying them may lead to new fundamental discoveries about actin and its role in healthy cell function.

The goal of this thesis is to set up a polarisation-resolved super-resolution fluorescence microscope to study these actin microstructures.	Microscopy -- imaging structures at microscopic scales -- is a wide and varied field. It is widely accepted to have begun in the sixteen-hundreds, with Antoni van Leeuwenhoek's discovery of bacteria and other single-celled organisms \cite{VanZuylen1981}. Over time, the resolution of microscopes strongly improved, and as lenses got better and better, they stopped being the resolution bottleneck. In 1873, Ernst Abbe determined that the best possible focus that a microscope can reach is limited by the wavelength of the light used \cite{Abbe1873}. This meant that microscopes of the time were limited to a resolution of roughly 200~nm (assuming focused blue light). It was long believed that the Abbe limit was a fundamental limit of nature, and that the only way around it was to use light of a different wavelength. This is one of the reasons for the development of electron microscopes \cite{Smith2008}, as quantum mechanics explains that accelerated electrons have a much shorter wavelength than visible light. 

However, using visible light for microscopy has some major advantages that electron microscopy cannot provide. Unlike other methods, it is able to tag specific protein species and other relevant molecules in the cell with a fluorescent label. For that reason, fluorescence microscopy has been an invaluable tool in modern biology \cite{Danial2016}.  There are thousands of small organic fluorescent labels, and more are being developed \cite{Zhang2002, Resch-Genger2008}. The introduction of GFP and other fluorescent proteins was a remarkable development in the field, as labels can now be genetically fused to proteins of interest \cite{Shaner2005, Matlashov2020}. In the year 2020 alone, over twenty thousand papers indexed in the PubMed database mention fluorescence in the title or abstract.

Fluorescence microscopy is not only capable of reporting on the location of a fluorophore, but also on its orientation. This is because fluorophores have a transition dipole moment; they will not absorb light that is polarised perpendicular to that dipole, and will only emit radiation polarised parallel to it. Therefore, fluorescence emission anisotropy is related to the dipole moment, which is determined by the orientation of the molecule that the fluorophore is bound to.

The group in which I conducted this thesis project, led by Jonas Tegenfeldt, already owned a stimulated emission depletion (STED) microscope, capable of performing fluorescence microscopy below the Abbe limit. It is fitted with some extra optics to enable polarisation microscopy by the supplier (Abberior GmbH, Germany), but as they had never done STED and polarisation microscopy at the same time, these capabilities were untested at the start of my thesis project. Therefore, the main goal of this thesis was to test the setup and properly calibrate it for polarisation microscopy experiments. Most of my learnings and contributions will therefore be found in the Background and Methods sections of this report. Along the way, we also came up with the idea of applying the working principle of STED in the polarisation domain and invented a microscopy protocol we call polarisation-resolved stimulated emission depletion microscopy (pSTED).