% !TeX spellcheck = en_GB
\chapter{Acknowledgements}

Many thanks to Jason and Jonas for their great supervision. You have taught me so much, both academically and personally, and I feel extremely lucky that I got to do my thesis project with you guys. Thanks also to the others in the Tegenfeldt group: Elham, Esra, Oskar and Bao, for the good vibes and the occasional helping hand.

I am grateful to our collaborators in the Nordenfelt and Swaminathan groups, particularly to Valeriia, Swathi and Oscar who provided the samples I used throughout my project. To the polarisation community in Lund at large: thank you. I am grateful for the opportunity to organise our discussion sessions, which turned out the be instrumental to several parts of my thesis project.

A special mention goes out to Carl Troein. He has not been directly involved in this project, but as my programme coordinator, he has helped me navigate a new university and has helped me out a ton with getting a summer internship, which turned out to be a stepping stone to the next challenge in my life.

I want to thank my friends and family for the continued support they gave me during this project, and simply for being in my life. Finally, I want to thank the Swedish public in general. I am extremely grateful for the opportunity to live in and discover this country for the past two years. If there's anything I've learnt from my time here, it's this: \emph{det finns inget dåligt väder, bara dåliga kläder}.

\bigskip

\noindent Jonas, I promised you I'd share my star recipes, so here they are.

\section*{Misir Wat}

\paragraph{Ingredients}
\begin{itemize}
	\item 4 tablespoons butter (or niter kibber, that's even better)
	\item 1 large yellow onion, very finely diced
	\item 3 cloves garlic, finely minced
	\item 1 tomato, very finely chopped
	\item 3 tablespoons tomato paste
	\item 2 tablespoons berbere , divided (you can get this at African Daily Market in Lund)
	\item 1 cup red lentils, rinsed
	\item 2.5 cups of vegetable stock
	\item 1 teaspoon salt
\end{itemize}

\paragraph{Preparation}
Melt 3 tablespoons of the butter in a medium stock pot.  Add the onions and cook over medium-high heat for 8-10 minutes until golden brown.  
Add the garlic, tomatoes, tomato paste and 1 tablespoon of the berbere and cook for 5-7 minutes. Reduce the heat if needed to prevent burning.
Add the broth and salt, bring it to a boil, reduce the heat to low and cover and simmer the lentils, stirring occasionally, for 40 minutes (adding more broth if needed) or until the lentils are soft.
Stir in the remaining tablespoon of butter and berbere. Simmer for a couple more minutes. Add salt to taste.
Serve with Ethiopian injera, sourdough bread, or rice.

This recipe is from \url{https://www.daringgourmet.com/misir-wat-ethiopian-spiced-red-lentils/}.

\section*{Ginger Beer}

To start a ginger bug, combine 0.5 litres of water, 20 g of ginger (chopped or grated, but not peeled), and 28 g sugar in a glass pot. Cover it with a clean towel and keep it at room temperature. Until it becomes fizzy, add 20 g of ginger and 30 g of sugar every day.

Then, combine 2 L of water, 200 g of sugar, and 75 g of ginger in a pot. Bring it to a boil, then reduce the heat and simmer for 5 to 8 minutes. Take it off the heat and let it cool down to room temperature. Add 100 g of strained ginger bug, and optionally some spices and the juice of three lemons. Divide over some airtight bottles and keep those at room temperature. Burp them daily by opening the cap to release built-up gas for about a week until the flavour is right and they are very carbonated. Then store them in the fridge for at least a couple of hours and enjoy!
