% !TeX spellcheck = en_GB
\chapter{Abstract}

This project aims to implement polarisation microscopy on the Tegenfeldt STED (stimulated emission depletion) microscope. STED microscopy is a targeted optical super-resolution method that can attain sub-diffraction resolution using visible light. This is now complemented with polarisation microscopy, which can measure the orientation of a fluorophore and, by extension, the molecule it is bound to. The current setup is one of the first systems in the world to combine STED and polarisation, but the its polarisation microscopy capability has never been tested.

In this thesis, I have characterised the effect of the various optical components in the microscope on the polarisation state of light and how they should be calibrated to perform several variations of polarisation microscopy. In the process, we have also developed a method that applies the operating principle of STED to increase resolution in the polarisation domain, which we call pSTED (polarisation-resolved stimulated emission depletion). The preliminary results are promising, but more work is required to demonstrate its potential as a new and innovative microscopy method.

All of the above methods have been applied to biological samples of human cell lines in which the actin cytoskeleton was fluorescently stained. The cells in these samples have been exposed to Yersinia bacteria. Pathogenic members of the Yersinia genus, of which the plague-causing \emph{Y.~pestis} is a member, destroy the actin cytoskeleton. In the absence of large actin fibres, it has been found that actin can form self-organising patterns on the microscale. This discovery opened up an exciting line of research where polarisation microscopy will be highly applicable.