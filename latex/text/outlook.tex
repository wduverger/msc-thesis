% !TeX spellcheck = en_GB

\chapter{Outlook}

Firstly, polarisation microscopy on the Tegenfeldt STED microscope has been achieved, and with that the main goal of this project. It is now possible to perform polarisation microscopy at confocal resolution and overlay the structure found by non-polarised STED micrographs taken on the same setup. This is already a great result. It was also confirmed that SiR-actin is a probe that is suitable for the setup: it is excited at 640~nm, it is efficiently depleted by the 775~nm laser and its bond to actin is sufficiently rigid for polarisation microscopy.

There is some work left before results on biological specimens are ready to be published, however. Firstly, the calibrations of the microscope can be improved. The 640~nm laser is not as well polarised as the 561~nm one is, despite going through a similar set of optics. The circularity of the STED polarisation can also be improved. Improving the quality of the laser polarisations would make polarisation measurements easier and more accurate. Secondly, the waveplates of both excitation lasers rotate more than necessary when the laser polarisation is rotated through the software. Improving this point should increase the speed of acquisition and the polarisation quality. Thirdly, a quantitative understanding of the depolarising effects of different components is needed, instead of simply a qualitative one. This knowledge may then be used for correcting and interpreting the raw data. For instance, a sample could be imaged in this setup and in another lab and compare the results quantitatively. If they agree, the setup is ready to produce publication-quality polarisation microscopy results.

Secondly, there is some other ground to be gained on the conventional polarisation microscopy front: in the detection waveplates. Once these are properly aligned, it becomes possible to use a polarising beam splitter in the detection pathway, which gives twice the amount of orientational information in a single acquisition. This is great for experiments that are limited by the photon budget: very sensitive samples can then be imaged with polarisation at confocal resolution, or less sensitive ones could be imaged at STED resolution. Therefore, the calibration of the detection waveplates should be tuned and measurement protocols that include them or a PBS can be developed. And if it is impossible to find exact alignment, then maybe it is possible to correct for that fact after acquisition.

Thirdly, the pSTED effort should be continued. The results so far are promising, but have not yet demonstrated the full potential of this method. This is due in part to finding the right balance between angular resolution and photodamage due to the increased depletion power. That problem is exacerbated by the fact that having a higher angular resolution requires acquiring more frames on top of the higher depletion power in every frame. Given the promising preliminary evidence, imaging a fresh sample could allow for polarisation images of higher quality and reveal the polarisation signature of actin micropatterns. On the more practical side of things, there are two obvious improvements that can be made: running the cabling to the HWP rotator stage through the back of the optical box instead of crushing under the lid of the optical enclosure and integrating the HWP into the microscope control software to automate pSTED acquisitions.

\todo{discuss why dropping the donut can be worth it.}

Finally, the setup offers some possibilities that have not yet been discussed in this thesis. There are many methods in which fluorescence polarisation can be added. Using the fluorescence lifetime imaging (FLIM) module, it should be possible to quantify the rotational diffusion of a fluorophore by assessing how its polarisation signal changes on a timescale of nanoseconds. Likewise, it should be possible to quantify fluorescence resonant energy transfer (FRET) between fluorophores by measuring the depolarisation effect the sample has on incoming light. Another opportunity is to apply these results to other research programmes in the Tegenfeldt group: polarisation microscopy should be able to quantify the degree of polarisation of DNA flowing through a microfluidic device, such as a DLD (deterministic lateral displacement) chip.